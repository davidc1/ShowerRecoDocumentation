
\section{EM Shower Reconstruction}
\label{sec:Reco}


In order to measure dE/dx correctly, it is extremely important to precisely determine the start point of the shower. 
This is due to two reasons - firstly, it is crucial to capture exactly the first couple of centimeters before the actual 
start of shower development, because at that point the electron and photon populations 
become indistinguishable. Secondly, the starting point is used in estimating the 
2D angle in each view, which is then translated to the 3D axis of the shower.

The proper determination of the 3D axis is important particularly to obtain 
agreement between collection and induction 
planes.  If the 3D axis is calculated incorrectly, the projected distance the particle travels 'dx' will be wrong
leading to errors in the dE/dx calculation. In such cases, the reconstructed dE/dx in the induction and collection planes will be significantly different
allowing the exclusion of mis-reconstructed events. 
%into the collection and 
%induction planes can cause the measure of dx in the dE/dx calculation to be 
%incorrect.  In particular, an incorrect 3D axis will typically cause the dx 
%between hits in one plane to be inflated, while in the other plane they are 
%decreased.  Because of this, a loose cut on the agreement between collection and 
%induction plane measures of dE/dx is included in the final analysis.


The reconstruction of the shower start poins and 3D axis is performed using the LArSoft software package \cite{LArSoft}, developed and shared by all experiments in the US 
LArTPC R\&D program. 
%In this framework, the ADC waveforms coming from each channel processed to 
%remove the electronics response and field shapes, especially the bi-modal 
%induction plane field shape. This process results in very similar waveforms for 
%both induction and collection planes allowing for their identical treatment. 

The events acquired in the detector are processed in order to remove the 
electronics and field-shape coherent noise using a FFT-based deconvolution 
kernel on a wire-by-wire basis. A peak finding algorithm is then used to find 
charge depositions on each wire, reconstructed as hits. The integral of 
the charge in each hit was used to calculate the charge $dQ$ using an  
(ADC$\times$Timetick)/Coulomb conversion constant. These constants are obtained 
for each wire separately, using through-going muon events in a way analogous to 
\cite{TGMuon}.

% Clustering of the Hit objects was performed in two ways - an automated 
%clustering algorithm is used first to connect the hits into associated groups in 
%each view. These automated clusters are then fed into a post-clustering which 
%calculates parameters like start points, angle in each view and the density of 
%hits, along the shower axis,the eigenvalue of the first Principal Component and 
%Charge RMS, the latter three being used for shower-track separation variables. 



Once a sample of shower candidates was selected, the hits for the candidate 
showers were manually assembled into clusters using a handscanning tool and fed into a 
shower-reconstruction algorithm. This allowed the refinement of the start point 
and direction in each plane for events with busy topologies.
% , where small tracks would be incorporated into the shower cluster when using 
%the automated reconstruction.
The 2D parameters, including the start point and 2D cluster axis, were used to calculate the 
3D axis of the electromagnetic shower object using an iterative algorithm.  \textit{For 
the start point, a guess based upon the two 2D start points is made for the 3D 
start point, and the start point in 3D is then projected into both planes.  The 
summed error across the planes provides information to adjust the 3D start 
point, and this process of project $\rightarrow$ calculate error $\rightarrow$ 
adjust 3D start point is continued until the 3D start point converges.}  
The 3D start point location can also be calculated using the start point 
locations obtained from the hand scan.  This allows a cross check between the 
reconstructed vertex from shower reconstruction and the hand scan.  For this 
analysis, a cut is made at 2cm between the two vertices to remove events that 
are poorly reconstructed. 
%This cut removes only 1 electron event and 2 photon events, however.

Similar to the 3D start point, the 3D axis is computed using an iterative 
projection matching algorithm.  A trigonometric formula is used to compute an 
approximate 3D axis based on the angle of each shower in the collection and 
induction plane.  The 3D axis is then projected into each plane, and the slope 
(in 2D) is compared against the slope of the electromagnetic showers in each 
plane.  Based upon the quality of the match between the projection and the 2D 
slopes, the 3D axis is adjusted until the best fit is obtained.

For the calorimetric separation of electrons and photons to succeed, the dE/dx 
metric must be well reconstructed. %For each the dx component of this measure is 
%the distance between observed depositions of energy in the detector.  
As the charge depositions are measured discretely in 2D 
in each of the wire planes we use the 3D 
axis of the shower to calculate an ``effective'' wire pitch between hits.  This 
effective pitch is, in other words, the real distance in the TPC that a particle 
travels between its two projections (hits) on adjacent wires - dx. 



An effective cross-check of this sample of events is the distribution of every dE/dx deposition 
measured, from all the events in the selected sample, presented in 
Fig.~\ref{fig:electron_landau} for electrons and Fig.~\ref{fig:photon_landau} 
for photons.  A Landau distribution, convolved with a Gaussian, is fit to the 
electron sample while a sum of two Gaussian-convolved Landaus is fit to the 
photon sample.  The good agreement between data and expectations gives a good indication of the .


\begin{figure}[tb]
   \centering
   
\includegraphics[width=0.45\textwidth]{figures/electron_landau_collection.png}
   \includegraphics[width=0.45\textwidth]{figures/electron_landau_induction.png}
   \caption{On the left is shown the distribution of dE/dx for all hits 
considered in the calorimetric separation for electrons, in the collection 
plane.  On the right is the same set of events using the induction plane.  
Unlike the aggregate measure of dE/dx, this plot uses the Modified Box Model to 
accurately capture the physics of highly ionizing fluctuations.  The parameters 
shown represent the best fit values of the fitted Gaussian-convolved Landau 
distributions.}
   \label{fig:electron_landau}
 \end{figure} 


\begin{figure}[tb]
   \centering
   \includegraphics[width=0.45\textwidth]{figures/photon_landau_collection.png}
   \includegraphics[width=0.45\textwidth]{figures/photon_landau_induction.png}
   \caption{On the left is shown the distribution of dE/dx for all hits 
considered in the calorimetric separation for photons, in the collection plane.  
On the right is the same set of events using the induction plane.  Unlike the 
aggregate measure of dE/dx, this plot uses the Modified Box Model to accurately 
capture the physics of highly ionizing fluctuations.  The parameters shown 
represent the best fit values of the fitted Gaussian-convolved Landau 
distributions, using two Gaussian-convolved Landau distributions to model the 
Compton scattering peak in the photon sample.  For the fit, the two 
distributions were constrained to have a the Landau most probable value be in a 
ratio of 1:2.}
   \label{fig:photon_landau}
 \end{figure} 



% Fig. \ref{fig:reco} shows the reconstruction quality of the 3D shower starting 
%points and the final 3D angles describing the shower axis, using the fully 
%automated reconstruction on single showers. 


% This in turn allowed calculating the true charge deposition per unit length - 
%$dQ/dx$ which was used to correct for quenching, via the Birks formula 
%\cite{Birks}. Before applying the quenching correction the charge was scaled to 
%compensate for losses due to impurities in the liquid argon, via the lifetime 
%correction\cite{Bruce}. 
