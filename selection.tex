
\section{Event Selection}
\label{sec:Selection}

In order to measure the separation of electron-like events from photon-like events  
a sub-sample of ArgoNeuT events that contained EM-showers was first isolated. This sub-sample was later used to select well defined electron and photon events
used to calculate the final separation.

% electrons and photons in a way that provides 
% excellent purity in each respective sample.  For this analysis, the samples were 
% achieved by applying a topological based cut described in 
% \ref{sec:topology_cut}.  Prior to this step, however, a sample of 
% electromagnetic showers must be isolated from the data.

Selecting the sub-sample of electromagnetic showers was automated and based on the information gained 
from the 2-dimensional clusters of charge depositions (hits). In the first step 
 empty events and those with only track like particles in them were removed from the sample using 
 an automated filter. This filter looked at two-dimensional clusters of hits made with the LArSoft package 
and calculated several parameters of these clusters to differentiate between track-like and shower-like clusters 
\cite{LArSoft}. The metrics that performed best in separating the two event populations were the modified hit density (MHD) 
where a large number of hits per unit of length of the cluster axis suggests a developing EM-shower and the first eigenvalue
obtained from Principal Component Analysis PCE, here a value close to 1 suggests track-like clusters, as they are naturally aligned along
the principal axis.
Fig. \ref{fig:separation} shows these separation 
parameters obtained using a Monte Carlo Simulation of single electrons as a prototype for electromagnetic showers, and 
single muons and protons as a prototype for tracks. The selection cuts defined to choose shower-like clusters were:
MHD > xx, PCE < yy. An additional requirement was that a shower-like cluster in on plane should correspond 
to an analagous cluster in the second plane. This removed spurious tags.


\begin{figure}[h]
\centering
\includegraphics[width=0.4\textwidth]{figures/trvssh_modhitdens.png}
\includegraphics[width=0.4\textwidth]{figures/trvssh_eigen_princ.png}
\caption{\label{fig:separation} ``Modified hit density'' and Principal Component 
Eigenvalue calculated for single electron showers(red) and muon track(blue).}
\end{figure}

Finally, an additional set of cuts was applied using all of the hits in a single view in 
an event as a single cluster. These cuts served to remove high-energy Deep 
Inelastic Scatter events and/or cosmic events which resulted in too much total 
charge in an event. 

This procedure resulted in sample of ArgoNeuT events that contained mostly EM-shower events, from which the final electron and photon samples were selected.

\subsection{Topological Separation of Electrons and Photons}
\label{sec:topology_cut}

When a photon is produced in an interaction in argon, it will travel some 
distance before it interacts with the argon and induces an electromagnetic 
shower.  Because photons are neutral particles, they do not ionize the argon 
until they interact with it via Compton Scattering or Pair Production channels, 
and thus there is a visible gap between the origin of the photon and the start 
of the electromagnetic shower. Unless there is other activity in the detector at the location of photon 
production, the gap is impossible to detect. Therefore, we have chosen to define two types of events as photon 
candidates: electromagnetic showers pointing back to charged particle activity 
at a vertex with hadronic interaction, and Neutral Current $\pi^0$ events where two 
electromagnetic showers project back to a common point.  In the second case, 
hadronic activity at the vertex was possible but not required, and both 
electromagnetic showers were used in the analysis.  Examples of both types of 
photon interactions are in Fig. \ref{fig:photons}.

\begin{figure}[h]
\centering
\includegraphics[width=0.455\textwidth]{figures/photon_R644_E11051_ind.png}
\includegraphics[width=0.455\textwidth]{figures/photon_R825_E12481_ind.png}
\includegraphics[width=0.455\textwidth]{figures/photon_R644_E11051_coll.png}
\includegraphics[width=0.455\textwidth]{figures/photon_R825_E12481_coll.png}
\caption{\label{fig:photons} Examples of photon candidate events.  The top row 
are the induction views and the bottom row are the collection views of two events. 
In both cases, the key identifying feature is the gap between 
the showers and the other activity to which they point backwards.}
\end{figure}

For a sample of electrons, this analysis targeted electron neutrino events as 
the electron shower candidates.  To ensure a high purity, an electromagnetic shower 
 was tagged as an electron only in events that also exhibited hadronic activity at the 
vertex {\em without} the presence of a gap between the shower and other 
particles.  In addition, a filter was applied to reject events where muons 
in the MINOS nead detector projected backwards into the ArgoNeuT TPC.  This 
ensures that the contamination from $\nu_\mu$ Charged Current events with high 
Bremsstrahlung activity is very low.  Examples of electron candidate showers are 
seen in Fig. \ref{fig:electrons}.

\begin{figure}[h]
\centering
\includegraphics[width=0.455\textwidth]{figures/electron_R775_E8598_ind.png}
\includegraphics[width=0.455\textwidth]{figures/electron_R827_E29437_ind.png}
\includegraphics[width=0.455\textwidth]{figures/electron_R775_E8598_coll.png}
\includegraphics[width=0.455\textwidth]{figures/electron_R827_E29437_coll.png}
\caption{\label{fig:electrons} Examples of electron candidate events.  The top row 
are the induction views and the bottom row are the collection views of two events. 
In both cases, there is no observable gap between the 
shower and the hadronic activity.}
\end{figure}

% Electron-candidate events were chosen as EM-Showers with no visible gap to the 
%vertex, whereas gamma candidate events were chosen as EM-Showers with a visible 
%gap to an existing interaction vertex. Additionally, for electron events it was 
%required that no track in the event be matched to a muon track in the MINOS Near 
%Detector. Fig. \ref{fig:events} shows an example of an electron and gamma event. 
%EM-Showers with no vertex activity were discarded and not used for this 
%analysis. The hand-scan was also used to define more precise clusters including 
%a seeding point for the start point determination. 

In total, 52 electron candidate showers and 274 photon candidate showers were 
selected for this analysis.


% In total XX electron-candidate events and YY gamma-candidate were chosen for 
%this analysis.


% The purity of the chosen samples was estimated by plotting the charge from the 
%first hits from all of the events in each sample and comparing them with 
%templates coming from Monte-Carlo reconstructed single electron and gamma 
%showers. 
% In the case of the gamma events, the samples are in reasonable agreement, 
%including the small peak at MIP energies, which is attributed to Compton 
%scatters, which is a probable method of interaction for photons at lower 
%energies. In the case of electrons, there is a small excess of energy 
%depositions around the energy expected from the gamma peak sample. This is 
%attributed to gammas converting near the vertex, that were mistaken for an 
%electron event. Using a likelihood fit, the contamination of such events was 
%estimated at 10\% of the electron sample - see Fig. \ref{fig:likelihood}.


% \begin{figure}[h]
% \centering
% \includegraphics[width=0.45\textwidth]{landau_electons_final.png}
% \includegraphics[width=0.445\textwidth]{landau_gammas_final.png}
% \caption{\label{fig:landaus} The distribution of the charge from all hits in 
%the first 2.4 cm of the reconstructed shower for electrons (left) and gammas (right).  }
% \end{figure}